\lab{Algorithms}{Bezier Curves and B-Splines}{Splines}

\objective{Understand the basics of Bezier Curves, B-Splines and deBoor's Algorithm}

\begin{itemize}
\item Suppose that we wish to smoothly approximate a function. We can't do this with linear interpolants (the derivative is always discontinuous) as we did with tesselation. Here we will investigate a few simple ways to smoothly build curves.
\item One of the difficulties is that we need a method of intuitively manipulating curves (changing tangent values isn't very intuitive). We can use bezier curves because they can b e manipulated intuitively.
\item Explain control points, Bernstein polynomials. Explain that Bernstein polynomials are ill-conditioned.
\item Maybe some extra discussion of the Bernstein polynomials is appropriate here. They allow uniform approximation of any continuous function, but are not useful computationally.
\item Introduce DeBoor's Algorithm (in this context it's called deCasteljau's algorithm). Point out that stability is due to its foundation in convex combinations.
\end{itemize}
\begin{problem}
Implement deCaseljau's algorithm. Investigate speed and stability against known curves, and compare to bernstein basis.
\end{problem}

\begin{problem}
Build an animation showing how the algorithm works. This shouldn't be very hard, but is pretty instructive.
\end{problem}

\begin{problem}
Python: Implement interactive control points.
\end{problem}

\begin{itemize}
\item We can build upon the foundation of bezier curves using a generalization known as b-splines
\item These are essentially piecewise functions, where we guarantee a certain degree of continuity.
\item The central difference is we define knot points, between which the function is a polynomial of specified degree. If we define all of our knots to be at the endpoints of our interval we actually get a bezier curve.
\item Like bezier curves, we have control points, which allow us to manipulate the curve in an intuitive way.
\item Talk about the definition, note, once again, the bernstein polynomials that would make direct evaluation difficult.
\item Talk about DeBoor's algorithm, which allows, again for easy, stable evaluation.
\end{itemize}

\begin{problem}
Extend the problems from bezier curves to allow b spline input and output.
\end{problem}

\begin{problem}
Is there a way to use barycentric langrange interpolation here? Test it out... (more of a curiosity than a problem I guess...)
\end{problem}

\begin{problem}
B Splines cannot represent certain classes of common curves exactly, such as circles. Explain why this is: I don't understand it well, but it's stated in numerous locations. This will help motivate nurbs later.
\end{problem}
