\lab{Applications}{Non-uniform Rational B-Spline}{NURBS}

\objective{Understand the basics of Non-uniform rational b-splines.}

{\bf Outline:}

\begin{itemize}
\item Explain some of the limitations of B-Splines (which we discussed previously. This includes lack of continuity of curvature (cubic B-Splines may have discontinuous curvature at knot points). Another major limitation is inability to represent conic sections (such as circles).
\item Explain that making the splines rational gives us greater flexibility, particularly in representing conic sections.
\item Note that these applications are useful in modelling and computer graphics
\item Example: a circle. Note that the curve doesn't travel uniformily around the circle. This makes sense, otherwise we could describe trig functions exactly using a rational function.
\item Note that we can use an adapted form of deBoor's Algorithm
\end{itemize}

\begin{problem}
Code up deBoor's to evaluate NURBS curves. This may generalize an earlier exercise.
\end{problem}

\begin{problem}
Use both NURBS and non-rational B-Splines to approximate an ellipse. How close are the two? Which one is computationally faster? Easier for a user?
\end{problem}

\begin{itemize}
\item Discuss Nurbs surfaces. Explain that much of 3-D modelling is based upon this. The extension for the math is not terribly difficult.
\item Also recall useful properties of B-Splines that are retained in rational form (invariant under affine transformations).
\end{itemize}

\begin{problem}
Create a NURBS sphere. Show how the surface is not parametrized uniformily.
\end{problem}
