\lab{Applications}{Variance Reduction Techniques}{Variance Reduction Techniques}

\objective{Explain how to apply variance reduction techniques to MC integration}

{\bf Outline:}
\begin{itemize}
\item Intro: Variance Reduction is about improving the variance of estimates of ``simulations''. This technique is used in statistics, system theory, etc. However, here we're just talking about an application to MC integration.
\item Toy example: pick something that varies only in one region, maybe tanh.
\item Explain how uniform sampling is not the most effective way to gather information.
\item Explain a little bit about how sampling correctly minimizes the variance of the error term. Wikipedia has a pretty good explanation in the VEGAS algorithm section. Basically if we pick random variables by the distribution $|f|/I(|f|)$ then we get zero variance in our error estimate.
\item Explain how the VEGAS algorithm tries to do this.
\end{itemize}

\begin{problem}
Estimate integral of $tanh$ using the VEGAS algorithm. Compare with the actual answer. Compare convergence speed with the naive MC integration technique.
\end{problem}

\begin{problem}
Try the VEGAS algorithm on the higher-dimensional problem used in MC integration section. Compare performance.
\end{problem}
