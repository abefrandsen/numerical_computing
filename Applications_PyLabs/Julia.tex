\lab{Applications}{Julia Sets and Basins of Attraction}{Julia Sets}

\objective{Understand definition of basin of attraction.  Basic understanding of Julia Sets.}

\begin{itemize}
\item Recall Newton's method
\item Different seed values for Newton yield different roots.
\item Definition of basin of attraction.
\end{itemize}

\begin{problem}
Draw a quadratic function and draw/color-code the basins of attraction on the x-axis.
\end{problem}
\begin{problem}
Do the same thing for a cubic function, i.e. $x(x-1)(x+1)$.  Increase resolution, examine basins.  What is going on here?!
\end{problem}

\begin{itemize}
\item Quick overview of complex functions.  Maybe make a shout out to analytic = nice?
\item Definition of orbits of complex functions
\item attractors/repellers, fixed points etc.
\item simple definition of julia set
\item We will be looking at functions that look like $f(z)  = z^2 - c$
\end{itemize}

\begin{problem}
If they have had complex analysis, throw a theory question in here about fixed points/attractors.
\end{problem}
\begin{problem}
Throw down a graph of the Julia set of a function.  This could be broken into several parts.  Begin with graphing the Julia set for $f(z) = z^2$ and $f(x) = z^2 - 2$.  This are well behaved.  Now throw some strange, complex values for $c$ that are between 0 and 2.  Crazy!
\end{problem}
\begin{problem}
(This could be a continuation of the previous problem)  Increase resolution in a specific area.  What happens? Where does it stop?  It doesn't!
\end{problem}

\begin{itemize}
\item Perhaps talk briefly about chaos and such, but this lab really should be about the pretty pictures.
\end{itemize}




