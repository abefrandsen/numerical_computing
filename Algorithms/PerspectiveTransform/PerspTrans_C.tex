\lab{Algorithms}{Perspective Transform}{Perspective Transform}

\objective{Understand the Perspective Transform in three dimensions.}

{\bf Outline:}

\begin{itemize}
\item Explain the physics of a pinhole camera. Explain that mathematically the human eye is similar.
\item We can model any type of camera in a similar manner(we usually just move the focal plane to the other side of the aperture so that the image on the focal plane is right-side up).
\item This allows us to render what a viewer will really see. Used widely in computer graphics, graphic design etc. (in art/graphic design it's related to train tracks meeting at infinity).
\item The math involved: Homogeneous coordinates (let you differentiate between vectors and points).
\item Generating rotation matrices. This uses some group theory and matrix exponentials.
\item Explain full-blown Perspective Transform.
\end{itemize}

\begin{problem}
Write a function that accepts a camera location, a viewing direction, a distance to the focal plane and a matrix of points$(n\times 3)$. Have the function return the points as seen on the focal plane (a $(n\times 2)$ matrix).
\end{problem}

\begin{problem}
Discretize some shape (either the MATLAB L or peaks or the Utah Teapot) and render it using the perspective transform. Maybe create an animated gif flyby?
\end{problem}

