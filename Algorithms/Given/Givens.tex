\lab{Algorithms}{Givens rotations}{Givens rotations}
\objective{Use orthogonal transformations to perform QR decomposition.}

The matrix $\begin{pmatrix} cos(\theta) & -sin(\theta) \\ sin(\theta) & cos(\theta) \end{pmatrix}$ rotates a vector counterclockwise by $\theta$. Given a vector $x = \begin{pmatrix} a \\ b \end{pmatrix}$, we can rotate $x$ into the range of $e_1$ by choosing the correct $\theta$. The problem is equivalent to solving for $c = cos(\theta)$ and $s = sin(\theta)$ in the system
\begin{equation}
\label{eq:Givens rotation system}
\begin{pmatrix} c & -s \\ s & c \end{pmatrix} \begin{pmatrix}a\\b\end{pmatrix} 
= \begin{pmatrix}r\\0\end{pmatrix}
\end{equation}
 In fact, it's not necessary to compute $\theta$; we solve for $c$ and $s$ directly. An obvious solution is $c = \frac{a}{\sqrt{a^2 + b^2}}$ and $s = \frac{-b}{\sqrt{a^2+b^2}}$.

\section*{Givens triangularization}

Like Householder, the Givens $QR$ algorithm, \emph{triangularizes} a matrix $A$ using \emph{orthogonal transformations}. The Householder $QR$ algorithm worked one column at a time; Givens works one element at a time. Each nonzero below the main diagonal can be zeroed out with a Givens rotation. If we solve for $c$ and $s$ as above, the matrix
\begin{equation*}
G(i,j,c,s) = 
\begin{pmatrix}
1   & \cdots & 0 & \cdots & 0 & \cdots & 0 \\
 \vdots & \ddots & \vdots &  & \vdots & & \vdots \\
0   & \cdots &    c   & \cdots &    -s   & \cdots &    0   \\
 \vdots &        & \vdots & \ddots & \vdots &        & \vdots \\
 0   & \cdots &   s   & \cdots &    c   & \cdots &    0   \\
\vdots &        & \vdots &        & \vdots & \ddots & \vdots \\
0   & \cdots &    0   & \cdots &    0   & \cdots &    1 
\end{pmatrix}
\end{equation*}
where
\[ \begin{array}{ll}
g_{i\, i} = c  & g_{j\, i}= -s   \\
g_{j\, j} = c &  g_{i\, j}= s  \\
\end{array}\]
rotates by $\theta$ in the $i,j$ plane. Also, it's easy to check that $G(i,j,c,s)$ is orthogonal. Left-multiplying $A$ by $G(i,j,c,s)$ zeroes out $A_{i\, j}$. For example, 

\[
\begin{array}{ccccccc}
\begin{pmatrix}
*&*&*\\
*&*&*\\
*&*&*
\end{pmatrix}
&
\underrightarrow{G(3,1)}
&\begin{pmatrix}
*&*&*\\
*&*&*\\
0&*&*
\end{pmatrix}
&
\underrightarrow{G(2,1)}
&\begin{pmatrix}
*&*&*\\
0&*&*\\
0&*&*
\end{pmatrix}
&
\underrightarrow{G(3,2)}
&\begin{pmatrix}
*&*&*\\
0&*&*\\
0&0&*
\end{pmatrix}
\end{array}
\]

The Givens QR method is often slower than Householder, because it works one element at a time. But it is faster than Householder when $A$ is sparse, and it is very parallelizable.

\begin{problem}
Write a script that uses Givens rotations to do $QR$ decomposition. By performing successive Givens rotations, triangularize $A$ to find $R$. You can find $Q$ using the chain of rotations.

To find each rotation, you will have to solve \eqref{eq:Givens rotation system} for $c$ and $s$. Let $b$ be the element you want to zero out, say $A_{i\,j}$. Then $a$ is in the same \emph{column} as $b$, and in the \emph{row} you want to rotate into (it's like we're squishing the whole vector into $a$'s spot.) You can choose $a$ to be the diagonal element above $b$.
\end{problem}

\begin{problem} 
Compare the MGS, Householder, and Givens algorithms for $QR$ decomposition on various matrices. Try different sizes and different levels of sparsity. Which is the fastest? Which is the most stable?
\end{problem}