\lab{Python}{Profiling Python Code}{Profiling}
\objective{Learn how to identify and remove bottlenecks in code.}
\label{lab:ProfilingCode}

You have undoubtedly written code that had less than stellar performance.
It would be really useful if we could see what parts of the code spent the most time executing.
We will introduce you to a number of ways to identify bottlenecks in your code.
A bottleneck is where the execution of a small part of the code slows down the overall execution speed.
The practice of locating these slower portions of code is called \emph{profiling}.
Before profiling, ensure that your code works as expected.
After profiling, we can attempt to optimize that section of the code so that it runs more efficiently.

\section*{When to optimize}
Trivial optimizations are of course encouraged.  
Examples of trivial optimizations are intelligent string concatenations, avoiding array and list copies when possible, etc.
Trivial optimizations should never be made at the expense of code readability.
If code readability is jeopardized, the optimization is not trivial.
Any nontrivial optimizations should not be done until after the code has been tested and bottlenecks identified.
Nontrivial optimizations, if performed incorrectly, may introduce errors into your program.
Donald Knuth, often referred to as the father of the analysis of algorithms, has written:
\begin{quote}
There is no doubt that the grail of efficiency leads to abuse.
Programmers waste enormous amounts of time thinking about, or worrying about, the speed of noncritical parts of their programs, and these attempts at efficiency actually have a strong negative impact when debugging and maintenance are considered.
We \emph{should} forget about small efficiencies, say about 97\% of the time: premature optimization is the root of all evil.
Yet we should not pass up our opportunities in that critical 3\%.
A good programmer will not be lulled into complacency by such reasoning, he will be wise to look carefully at the critical code; but only \emph{after} that code has been identified.
It is often a mistake to make a priori judgments about what parts of a program are really critical, since the universal experience of programmers who have been using measurement tools has been that their intuitive guesses fail.
\footnote{Knuth, Donald. \emph{Structured Programming with go to Statements}. Computing Surveys \textbf{6}(4), 1974, 261.}
\end{quote}
In other words, don't waste time making nontrivial optimizations to code that doesn't matter.
When your code runs acceptably fast, stop optimizing.

\section*{Profiling Code}
Python includes a code profiler in its standard library. 
The \li{cProfile} module tracks function calls and their execution while your code runs.
Understand that running your program with a profiler will slow down your program slightly, but times are still comparable since the profiler overhead is a constant factor.
The simplest way to profile your code is directly from the command line.
\begin{lstlisting}
python -m cProfile myscript.py
\end{lstlisting}
This will print the profile results directly to the console, but these results can also be saved to a file for later viewing by adding \texttt{-o <filename>}.
You can then analyze the results using the \li{pstats} module, also in the Python standard library.

You can also import the \li{cProfile} module and use it directly.
\begin{lstlisting}
import cProfile
def fac(n):
    if n == 1:
        return 1
    else:
        return fac(n-1)*fac(n-2)

cProfile.run('fib(100)')
\end{lstlisting}

IPython provides a nice interface to \li{cProfile} via the \li{\%prun} magic function.
\begin{lstlisting}
: \%prun fib(100)
\end{lstlisting}


\section*{What to Optimize}
There are several common optimization tricks that might help you determine where to start first.
\begin{itemize}
\item Avoid recomputing values unnecessarily
    \subitem Move expensive computations out of loops
    \subitem Avoid copying data structures
    \subitem Cache results
    \subitem Cache often used methods
\item Avoid nested loops
    \subitem Simplify inner loops
    \subitem Consider using array operations instead of looping
\item Use builtin functions or library functions implemented in C
\item Avoid excessive function calls
\item Use a more efficient algorithm
\item Use generators when possible
\item Simplify mathematical expressions where possible
\end{itemize}

Let's look at an example of optimizing a function that calculates the QR decomposition of a matrix.
\begin{lstlisting}
def qr1(X):
    Q = X.copy()
    R = np.zeros((X.shape[0], X.shape[1]))
    for i in range(X.shape[1]):
        R[i,i] = la.norm(Q[:,i])
        Q[:,i] = Q[:,i]/la.norm(Q[:,i])
        for j in range(i+1, X.shape[1]):
            R[i, j] = Q[:,j].dot(Q[:,i])
            Q[:,j] = Q[:,j] - Q[:,j].dot(Q[:,i]) * Q[:,i]
    return Q, R
\end{lstlisting}
We generate a random $100 \times 100$ 2D array for profiling.  The output of \li{cProfile} should look similar to below.
{\scriptsize
\begin{verbatim}
      11806 function calls in 0.097 seconds

Ordered by: internal time

ncalls  tottime  percall  cumtime  percall filename:lineno(function)
     1    0.070    0.070    0.097    0.097 qropt.py:48(qr1)
  9900    0.023    0.000    0.023    0.000 {method 'dot' of 'numpy.ndarray' objects}
   200    0.001    0.000    0.004    0.000 misc.py:11(norm)
   200    0.001    0.000    0.001    0.000 {method 'reduce' of 'numpy.ufunc' objects}
   200    0.001    0.000    0.003    0.000 function_base.py:527(asarray_chkfinite)
   200    0.000    0.000    0.000    0.000 {numpy.core.multiarray.array}
   200    0.000    0.000    0.001    0.000 _methods.py:28(_all)
   200    0.000    0.000    0.001    0.000 {method 'all' of 'numpy.ndarray' objects}
   200    0.000    0.000    0.000    0.000 numeric.py:252(asarray)
   101    0.000    0.000    0.000    0.000 {range}
   200    0.000    0.000    0.000    0.000 {getattr}
   200    0.000    0.000    0.000    0.000 {method 'get' of 'dict' objects}
     1    0.000    0.000    0.000    0.000 {numpy.core.multiarray.zeros}
     1    0.000    0.000    0.000    0.000 {method 'copy' of 'numpy.ndarray' objects}
     1    0.000    0.000    0.097    0.097 <string>:1(<module>)
     1    0.000    0.000    0.000    0.000 {method 'disable' of '_lsprof.Profiler' objects}
\end{verbatim}
}
Executing \texttt{qr1(a)} results in almost 12000 function calls.  
We then receive a listing of each function that was called.  In the far left column, we see how many times each function was called.
Next we see the total time spent in a given function.
The cumulative time, or \texttt{cumtime}, column measures how much time was spent in a given function including all calls to any other function.

The most obvious, and trivial, optimizations might be 
\begin{itemize}
\item Avoid recomputing \li{R[i,i]} in the outer loop and \li{R[i,j]} in the inner loop
\item Use \li{xrange} instead of \li{range}
\item Cache the size of the array outside the loop
\end{itemize}
\begin{lstlisting}
def qr2(X):
    Q = X.copy()
    R = np.zeros(X.shape)

    ncols = X.shape[1]
    for i in xrange(ncols):
        R[i,i] = la.norm(Q[:,i])
        Q[:,i] = Q[:,i]/R[i,i]
        for j in xrange(i+1, ncols):
            R[i,j] = Q[:,j].dot(Q[:,i])
            Q[:,j] = Q[:,j] - R[i,j] * Q[:,i]

    return Q, R
\end{lstlisting}
These small changes, especially the removing the recomputations, speed things up nicely.
{\scriptsize
\begin{verbatim}
      5855 function calls in 0.080 seconds

Ordered by: internal time

ncalls  tottime  percall  cumtime  percall filename:lineno(function)
     1    0.064    0.064    0.080    0.080 qropt.py:73(qr2)
  4950    0.013    0.000    0.013    0.000 {method 'dot' of 'numpy.ndarray' objects}
   100    0.001    0.000    0.002    0.000 misc.py:11(norm)
   100    0.001    0.000    0.001    0.000 {method 'reduce' of 'numpy.ufunc' objects}
   100    0.001    0.000    0.001    0.000 function_base.py:527(asarray_chkfinite)
   100    0.000    0.000    0.000    0.000 {numpy.core.multiarray.array}
   100    0.000    0.000    0.001    0.000 _methods.py:28(_all)
   100    0.000    0.000    0.000    0.000 numeric.py:252(asarray)
   100    0.000    0.000    0.001    0.000 {method 'all' of 'numpy.ndarray' objects}
   100    0.000    0.000    0.000    0.000 {getattr}
     1    0.000    0.000    0.000    0.000 {method 'copy' of 'numpy.ndarray' objects}
   100    0.000    0.000    0.000    0.000 {method 'get' of 'dict' objects}
     1    0.000    0.000    0.000    0.000 {numpy.core.multiarray.zeros}
     1    0.000    0.000    0.080    0.080 <string>:1(<module>)
     1    0.000    0.000    0.000    0.000 {method 'disable' of '_lsprof.Profiler' objects}
\end{verbatim}
}
Let's use a larger, $600 \times 600$, array to exaggerate the timing differences.
{\scriptsize
\begin{verbatim}
      185105 function calls in 5.612 seconds

Ordered by: internal time

ncalls  tottime  percall  cumtime  percall filename:lineno(function)
     1    4.253    4.253    5.612    5.612 qropt.py:74(qr2)
179700    1.335    0.000    1.335    0.000 {method 'dot' of 'numpy.ndarray' objects}
   600    0.009    0.000    0.022    0.000 misc.py:11(norm)
   600    0.006    0.000    0.013    0.000 function_base.py:527(asarray_chkfinite)
   600    0.004    0.000    0.004    0.000 {method 'reduce' of 'numpy.ufunc' objects}
   600    0.001    0.000    0.001    0.000 {numpy.core.multiarray.array}
   600    0.001    0.000    0.005    0.000 _methods.py:28(_all)
   600    0.001    0.000    0.002    0.000 numeric.py:252(asarray)
     1    0.001    0.001    0.001    0.001 {method 'copy' of 'numpy.ndarray' objects}
   600    0.001    0.000    0.005    0.000 {method 'all' of 'numpy.ndarray' objects}
     1    0.000    0.000    0.000    0.000 {numpy.core.multiarray.zeros}
   600    0.000    0.000    0.000    0.000 {getattr}
   600    0.000    0.000    0.000    0.000 {method 'get' of 'dict' objects}
     1    0.000    0.000    5.612    5.612 <string>:1(<module>)
     1    0.000    0.000    0.000    0.000 {method 'disable' of '_lsprof.Profiler' objects}
\end{verbatim}
}
This seems to about as fast as we can make it.  We cannot avoid the dot product or calculating the norm.
Most of the other calls are not made directly by our function.  
One option could be to try using \li{np.inner} instead of \li{np.dot}, but only makes it marginally faster.

Let's look at a non-trivial optimization.  We can see that we are doing a lot of column slicing with Q.
NumPy arrays are row major which means that the array is stored in memory by row.
This makes column slicing cumbersome because a column has exactly one element in each row, thus requiring NumPy
to jump to many different memory locations.  If we could formulate our algorithm to use rows instead of columns,
we can avoid the jumping around in memory and instead read a larger contiguous block.
We accomplish this by making \li{Q} a transposed copy of \li{X} and switching all of our column slices to row slices.
Be careful when you make the switch.
{\scriptsize
\begin{verbatim}
      185105 function calls in 1.402 seconds

Ordered by: internal time

ncalls  tottime  percall  cumtime  percall filename:lineno(function)
     1    1.086    1.086    1.402    1.402 qropt.py:5(QR)
179700    0.298    0.000    0.298    0.000 {numpy.core.multiarray.inner}
   600    0.005    0.000    0.015    0.000 misc.py:11(norm)
   600    0.004    0.000    0.004    0.000 {method 'reduce' of 'numpy.ufunc' objects}
   600    0.004    0.000    0.010    0.000 function_base.py:527(asarray_chkfinite)
     1    0.002    0.002    0.002    0.002 {method 'copy' of 'numpy.ndarray' objects}
   600    0.001    0.000    0.001    0.000 {numpy.core.multiarray.array}
   600    0.001    0.000    0.001    0.000 numeric.py:252(asarray)
   600    0.001    0.000    0.004    0.000 _methods.py:28(_all)
   600    0.000    0.000    0.005    0.000 {method 'all' of 'numpy.ndarray' objects}
     1    0.000    0.000    0.000    0.000 {numpy.core.multiarray.zeros}
   600    0.000    0.000    0.000    0.000 {getattr}
   600    0.000    0.000    0.000    0.000 {method 'get' of 'dict' objects}
     1    0.000    0.000    1.402    1.402 <string>:1(<module>)
     1    0.000    0.000    0.000    0.000 {method 'disable' of '_lsprof.Profiler' objects}
\end{verbatim}
}
The difference is immediately noticeable.  Remember that all of our optimizations thus far have been
to the way this QR algorithm is implemented.  Keep in mind that it is often far better to spend time optimizing the algorithm
than to optimize its implementation.  Even algebraic simplification of an algorithm can yield huge increases in speed.
One example of optimizing an algorithm is found in calculating Voronoi diagrams.  The algorithms that were commonly used were $O(n^2)$.
The attempts to optimize the implementation only reduced computation time by a constant factor.
Then in 1986, Steve Fortune published an algorithm that could calculate Voronoi diagrams on a plane in $O(n \log n)$.
This vastly improved the time needed to compute large Voronoi diagrams that were, up to that point in time, prohibitive to compute.

\begin{problem}
Profile and optimize your solutions to the following labs
\begin{itemize}
\item Givens
\item Householder
\item Image Segmentation
\end{itemize}
Your solution should include a before and an after profiling showing the overall optimization in each of your solutions.  It should also include a list of changes, the reasoning behind the changes, and the effect of the changes on runtime.
\end{problem}
