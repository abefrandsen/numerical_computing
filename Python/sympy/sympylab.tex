\lab{Python}{Symbolic Computation in Python}{Sympy}
\label{lab:Sympy}

\objective{Become familiar with some of the basic tools available in Sympy}

Python is good for more than just analysis of numerical data.
There are several packages available which allow symbolic computation in Python.
One such package is Sympy.
Sympy is designed to be a fully featured computer algebra system in Python.
An example of what we mean by "symbolic computation" is the following:
\begin{lstlisting}
import sympy as sy
x=sy.symbols('x')
sy.expand((x+1)**10)
\end{lstlisting}
which will return the following:
\begin{lstlisting}
x**10 + 10*x**9 + 45*x**8 + 120*x**7 + 210*x**6 + 252*x**5 + 210*x**4 + 120*x**3 + 45*x**2 + 10*x + 1
\end{lstlisting}
The idea is that this is a package that, when used properly, can do large amounts of algebra for you.
This can be incredibly useful in a wide variety of situations.
As you may have guessed, such packages generally have a wide variety of features. 
If you want to know how to do something, consider checking the documentation  at the Sympy \href{http://sympy.org/en/index.htmll}{website}.
This lab should teach you to do some basic symbolic manipulations in Sympy.
