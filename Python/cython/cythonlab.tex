\lab{Cython}{Cython}{Cython}

\objective{Learn the basics of using Cython to speed up heavy computation}

In Python, everything is an object.
This can make many computations very convenient, but it does come at a minor speed cost.
When we are doing lots of heavy computation, removing the interfaces to python objects can speed up programs significantly.
There are a variety of ways to do this.
Wrapping C and Fortran functions that are designed to work with python objects is one good way of speeding up heavy computation.
It is often easier to use Cython.
Cython is a language designed to allow easy interfacing between C and Python.
Cython is a superset of the Python language.
In theory, all valid Python code is also valid Cython code.
It is essentially just Python with some extra type declarations.
It also allows you to easily import some C functions.
Cython is, however, not run interactively.
Cython is written in a \li{.pyx} file which is then compiled to C, then compiled to machine code.
There are a variety of ways to import Cython functions and classes to Python.
The standard way is to make a python file containing a reference to the Cython file, then import that.
The Ipython \li{\%\%cython} cell magic command in the Ipython notebook will compile the cell as a Cython file and import all functions and classes declared there into your current namespace.
The library \li{pyximport} also allows you import many Cython files directly instead of having to use the extra header file.
Pyximport will work if the Cython library you are importing "does not require any extra C libraries or a special build setup."

\section*{typed for loops}

