\lab{Python}{Intro to pandas}{Intro to pandas}

In volumes 1 and 2, we solved data problems primarily using NumPy and SciPy.
We now turn our attention to a Python library that is more specifically built for data
analysis. Welcome to \emph{pandas}.

\section*{Data Structures in pandas}
Just as NumPy is built on the ndarray data structure suited for efficient scientific and numerical
computation, pandas is centered around a handful of core data structures custom built for data
analysis. These data structures include the Series, DataFrame, and Panel. We explore each in turn.

\subsection*{Series}
A Series is a one-dimensional array whose entries are labeled. The values of the array may be 
any data type, including integers, strings, or general Python objects. Further, the array
need not be homogeneous. That is, it can hold values of different data types. Together, the
array values are referred to as the data of the Series.

The labels must consist of hashable types. They are commonly integers or strings. 
Together, the labels are referred to as the index of the Series.

Thus, a Series consists of data and an index. The most basic way to initialize such an object
is as follows:
\begin{lstlisting}
>>> import pandas as pd
>>> s = pd.Series(data, index=index)
\end{lstlisting}
We needn't explicitly define the index. The default index is simply \li{np.arange(len(data))}.

For example, we can create a Series containing the integers from 9 down to 0:
\begin{lstlisting}
>>> s1 = pd.Series(range(9, -1, -1))
>>> s1.values    #the data
array([9, 8, 7, 6, 5, 4, 3, 2, 1, 0], dtype=int64)
>>> s1.index     #the labels
Int64Index([0, 1, 2, 3, 4, 5, 6, 7, 8, 9], dtype='int64')
>>> s1           #left column is index, right column is data
0    9
1    8
2    7
3    6
4    5
5    4
6    3
7    2
8    1
9    0
dtype: int64
\end{lstlisting}

Here is an example where we create customized labels:
\begin{lstlisting}
>>> data = np.random.randn(3)
>>> index = ['first', 'second', 'third']
>>> s2 = pd.Series(data, index=index)
>>> s2
first     1.661255
second   -0.033570
third    -2.185991
dtype: float64
\end{lstlisting}

We can create a Series having constant values in the following manner:
\begin{lstlisting}
>>> val = 4     #desired constant value of Series
>>> n = 6      #desired length of Series
>>> s3 = pd.Series(val, index=range(n))
>>> s3
0    4
1    4
2    4
3    4
4    4
5    4
dtype: int64
\end{lstlisting}

It is also possible to use a Python dictionary when creating a Series:
\begin{lstlisting}
>>> d = {'s1':93, 's2':95, 's3':87, 's4':82, 's5':94}
>>> s4 = pd.Series(d)
>>> s4
s1    93
s2    95
s3    87
s4    82
s5    94
dtype: int64
\end{lstlisting}
Note that we didn't need to specify the index: the keys of the dictionary are used as the index for the Series. 

\begin{problem}
Create the following pandas Series.

\begin{itemize}
\item Constant array with value -3, length 5. Labels should be the first five positive even integers.

\item Data is given by the dictionary \li{{'Bill':31, 'Sarah':28, 'Jane':34, 'Joe':26}}.
\end{itemize}
\end{problem}


\subsection*{DataFrame}

\subsection*{Panel}

\section*{Viewing Data}
Indexing.
Slice columns.
\li{head} and \li{tail} methods.
Elementary database operations (select, join)

\section*{Manipulating Data}
Cleaning data. Transposing. Sorting.

\section*{Analyzing Data}
\li{describe} method.

\section*{Data I/O}
Before we
