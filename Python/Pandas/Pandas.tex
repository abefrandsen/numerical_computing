\lab{Python}{Intro to pandas}{Intro to pandas}

In volumes 1 and 2, we solved data problems primarily using NumPy and SciPy.
We now turn our attention to a Python library that is more specifically built for data
analysis. Welcome to \emph{pandas}.

\section*{Data Structures in pandas}
Just as NumPy is built on the ndarray data structure suited for efficient scientific and numerical
computation, pandas is centered around a handful of core data structures custom built for data
analysis. These data structures include the Series, DataFrame, and Panel. We explore each in turn.

\subsection*{Series}
The Series is a one-dimensional array whose entries are labeled. The values of the array may be
any data type, including integers, strings, or general Python objects. Further, the array
need not be homogeneous. That is, it can hold values of different data types. Together, the
array values are referred to as the data of the Series.

The labels must consist of hashable types. They are commonly integers or strings.
Together, the labels are referred to as the index of the Series.

Thus, a Series consists of data and an index. The most basic way to initialize such an object
is as follows:
\begin{lstlisting}
>>> import pandas as pd
>>> s = pd.Series(data, index=index)
\end{lstlisting}
We needn't explicitly define the index. The default index is simply \li{np.arange(len(data))}.

For example, we can create a Series containing the integers from 9 down to 0:
\begin{lstlisting}
>>> s1 = pd.Series(range(9, -1, -1))
>>> s1.values    #the data
array([9, 8, 7, 6, 5, 4, 3, 2, 1, 0], dtype=int64)
>>> s1.index     #the labels
Int64Index([0, 1, 2, 3, 4, 5, 6, 7, 8, 9], dtype='int64')
>>> s1           #left column is index, right column is data
0    9
1    8
2    7
3    6
4    5
5    4
6    3
7    2
8    1
9    0
dtype: int64
\end{lstlisting}

Here is an example where we create customized labels:
\begin{lstlisting}
>>> data = np.random.randn(3)
>>> index = ['first', 'second', 'third']
>>> s2 = pd.Series(data, index=index)
>>> s2
first     1.661255
second   -0.033570
third    -2.185991
dtype: float64
\end{lstlisting}

We can create a Series having constant values in the following manner:
\begin{lstlisting}
>>> val = 4     #desired constant value of Series
>>> n = 6       #desired length of Series
>>> s3 = pd.Series(val, index=range(n))
>>> s3
0    4
1    4
2    4
3    4
4    4
5    4
dtype: int64
\end{lstlisting}

It is also possible to use a Python dict when creating a Series:
\begin{lstlisting}
>>> d = {'e1':93, 'e2':95, 'e3':87, 'e4':82, 'e5':94}
>>> s4 = pd.Series(d)
>>> s4
e1    93
e2    95
e3    87
e4    82
e5    94
dtype: int64
\end{lstlisting}
Note that we didn't need to specify the index: the keys of the dict are used as the index for the Series.

\begin{problem}
Create the following pandas Series.

\begin{itemize}
\item Constant array with value -3, length 5. Labels should be the first five positive even integers.

\item Data is given by the dict \{`Bill':31, `Sarah':28, `Jane':34, `Joe':26\}.
\end{itemize}
\end{problem}

As with the ndarray, we can slice a Series using the usual syntax:
\begin{lstlisting}
>>> s4[:3]
e1    93
e2    95
e3    87
dtype: int64
\end{lstlisting}
Notice that both the data and the index are sliced in this manner.
We can also input Series with numerical data into many NumPy functions.
\begin{lstlisting}
>>> np.log(s4)
e1    4.532599
e2    4.553877
e3    4.465908
e4    4.406719
e5    4.543295
dtype: float64
\end{lstlisting}

A Series also has similarities with the Python dict.
We can access and alter the data using the index.
\begin{lstlisting}
>>> s4['e3']
87
>>> s4['e3'] = 99
>>> s4['e3']
99
\end{lstlisting}

We finish off this section by considering elementary vectorized operations with Series.
\begin{lstlisting}
>>> x = pd.Series(np.random.randn(4), index=['a', 'b', 'c', 'd'])
>>> x
a   -0.924259
b    0.767422
c    0.399212
d    0.130365
dtype: float64
>>> y = pd.Series(np.random.randn(5), index=['a', 'b', 'd', 'e', 'f'])
>>> y
a   -0.708301
b   -2.214516
d   -2.352364
e    0.789419
f   -0.859482
dtype: float64
\end{lstlisting}
Much as with the NumPy array, we can perform basic arithmetic operations on the entries of
a Series without the use of a for-loop. For example, we square the elements of \li{x} as
follows:
\begin{lstlisting}
>>> x**2
a    0.854254
b    0.588937
c    0.159370
d    0.016995
dtype: float64
\end{lstlisting}
In some cases, the Series allows for even greater flexibility than the NumPy array. For example,
we are able to add Series \li{x} and \li{y}, even though they have different lengths and labels:
\begin{lstlisting}
>>> z = x+y
>>> z
a   -1.632559
b   -1.447093
c         NaN
d   -2.221999
e         NaN
f         NaN
dtype: float64
\end{lstlisting}
Notice that the index of \li{z} is the \emph{union} of the index of \li{x} and the index of \li{y}.
For the labels shared by both \li{x} and \li{y} (namely \li{a}, \li{b}, and \li{d}), the corresponding
value of \li{z} is just the sum of the entries of \li{x} and \li{y}. In all other cases, the value of 
\li{z} is \li{NaN}, which is the pandas type indicating a missing value. It is simple to omit missing 
values from a series:
\begin{lstlisting}
>>> z.dropna()
a   -1.632559
b   -1.447093
d   -2.221999
dtype: float64
\end{lstlisting}
\subsection*{DataFrame}
The DataFrame data structure is a two-dimensional generalization of the Series. It can be viewed
as a tabular structure with labeled rows and columns. The row labels are collectively called the
index, and the column labels are collectively called the columns. An individual columns in a 
DataFrame object is a Series.

There are many ways to initialize a DataFrame. In the following, we build a DataFrame out of a
dict of Series.
\begin{lstlisting}
>>> d = {1:x, 2:y}
>>> df1 = pd.DataFrame(d)
>>> df1
	        1	        2
a	-0.924259	-0.708301
b	 0.767422	-2.214516
c	 0.399212	      NaN
d	 0.130365	-2.352364
e	      NaN	 0.789419
f	      NaN	-0.859482
\end{lstlisting}
Note that the index of this DataFrame is the union of the index of Series \li{x} and that of Series \li{y}.
The columns are given by the keys of the dict \li{d}. Since \li{x} doesn't have a label \li{e}, the
value in row \li{e}, column \li{1} is \li{NaN}. This same reasoning explains the other missing values as well.
Note that if we take the first column of the DataFrame and drop the missing values, we recover the Series \li{x}:
\begin{lstlisting}
>>> x == df1[1].dropna()
a    True
b    True
c    True
d    True
dtype: bool
\end{lstlisting}

We can also initialize a DataFrame using a NumPy array, creating custom row and column labels:
\begin{lstlisting}
>>> data = np.random.random((3,4))
>>> pd.DataFrame(data, index=['A','B','C'], columns=range(1,5))

            1	        2	        3	        4
A	 0.065646	 0.968593	 0.593394	 0.750110
B	 0.803829	 0.662237	 0.200592	 0.137713
C	 0.288801	 0.956662	 0.817915	 0.951016
3 rows � 4 columns
\end{lstlisting}
As with Series, if we don't specify the index or columns, the default is \li{range(n)}, where \li{n} is either the number of 
rows or columns.

A DataFrame behaves in certain respects like a dict of Series, where each column label and the corresponding column form the key-value pairs.
We can access a desired column via its column label:
\begin{lstlisting}
>>> df1[2]
a   -0.708301
b   -2.214516
c         NaN
d   -2.352364
e    0.789419
f   -0.859482
Name: 2, dtype: float64
\end{lstlisting}
We can insert a new column or delete a column much as we would with a dict:
\begin{lstlisting}
>>> df1['product'] = df1[1] * df1[2]  #insert column containing the product of columns 1 and 2
>>> df1['constant'] = 'const'         #insert column containing constant data
>>> df1
            1	        2	  product	 constant
a	-0.924259	-0.708301	 0.654653	    const
b	 0.767422	-2.214516	-1.699469	    const
c	 0.399212	      NaN	      NaN	    const
d	 0.130365	-2.352364	-0.306666	    const
e	      NaN	 0.789419	      NaN	    const
f	      NaN	-0.859482	      NaN	    const
6 rows � 4 columns
>>> del df1['constant']
            1	        2	  product	
a	-0.924259	-0.708301	 0.654653	
b	 0.767422	-2.214516	-1.699469	
c	 0.399212	      NaN	      NaN	
d	 0.130365	-2.352364	-0.306666	
e	      NaN	 0.789419	      NaN	
f	      NaN	-0.859482	      NaN	
6 rows � 4 columns
\end{lstlisting}

We can also select specified rows, either using the row label, or its integer position:
\begin{lstlisting}
>>> df1.loc['b']        #select 2nd row via label
1          0.767422
2         -2.214516
product   -1.699469
Name: b, dtype: float64
>>> df1.iloc[1]         #select 2nd row via integer position
1          0.767422
2         -2.214516
product   -1.699469
Name: b, dtype: float64
\end{lstlisting}
We can slice rows much as we do with NumPy arrays:
\begin{lstlisting}
>>> df1[1:4]
	        1	        2	  product
b	 0.767422	-2.214516	-1.699469
c	 0.399212	      NaN	      NaN
d	 0.130365	-2.352364	-0.306666
3 rows � 3 columns
\end{lstlisting}
\subsection*{Panel}

\section*{Viewing Data}
Indexing.
Slice columns.
\li{head} and \li{tail} methods.
Elementary database operations (select, join)
Plotting.

\section*{Manipulating Data}
Cleaning data. Transposing. Sorting.

\section*{Analyzing Data}
\li{describe} method.
Have them read in financial data for several stocks. Then calculate means and variances of each. Find the stock with highest mean, lowest variance.
\section*{Data I/O}
Before we
