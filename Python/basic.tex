\lab{}{The Python Language}{The Python Language}
\label{Lab:Python}

In this lab we will focus on the basic constructs of the Python programming language.  This will serve as a basic introduction in using the core language.  Understanding this lab will be essential to your success in the following labs.

\section*{Starting out}
We will do most of our development using a Python shell called IPython.  It has many advanced features that will prove to be very useful to you later on.  Let's start with the quintessetial first program.

Start the Ipython interpreter.

\section*{Data types}
Python has several different way to store type of data.  In this section, we will explore all of Python's basic ways to store data.

\subsection*{Numbers}
Storing numbers is easy in Python.  Often, we don't have to worry about how to store numbers as Python is usually smart enough determine the best way to remember the numbers we give it.
\begin{lstlisting}[style=python]
: a = 1
: b = 0.5
: c = 1+2j
: d = True
: type(a)
int
: type(b)
float
: type(c)
complex
: type(d)
bool
\end{lstlisting}
Note that Python represents some numbers differently.

\subsection*{Containers}
Container types are ways to store several values at a time.  The main container types in Python are lists, strings, tuples, dictionaries, and sets.

\subsubsection*{Lists}
Lists are Python's workhorse datatype.  A mastery of the data type will be extremely useful in your future with the Python language.

A list is an ordered collection of arbitrary types.
\begin{lstlisting}[style=python]
: l = [1, 0.5, True, 1+2j]
: type(l)
list
\end{lstlisting}
Lists are indexed objects.  This means that individual objects can be retreived from the list by knowing its position.  The first object in a list always has index $0$.  Python also recognizes negative indexes and will count from the end of the list.  The index of $-1$ will always retreive the last object in the list.
\begin{lstlisting}[style=python]
: l[0]
1
: l[-1] == l[3]
True
\end{lstlisting}
We can also extract sublists very easily by \emph{slicing} a list.  The slicing syntax is \li{l[start:stop:step]} which will return a a list of the elements \li{i} such that \li{start <= i < stop
\begin{lstlisting}[style=python]
: l[1:]
[0.5, True, 1+2j]
: l[1:3]
[0.5, True]
: l[:2]
[1, 0.5]
: l[::2]
[1, True]
: l[::-1]
[1+2j, True, 0.5, 1]
\end{lstlisting}




\subsubsection*{Strings}

\subsubsection*{Tuples}

\subsubsection*{Dictionaries}

\subsubsection*{Sets}

\section*{Conditional Statements}

\section*{Loops}
\subsection*{for}
\subsection*{while}

\section*{Functions}

\section*{Scripts}

\section*{Libraries}