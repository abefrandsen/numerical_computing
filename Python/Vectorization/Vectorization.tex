\lab{Python}{Array Programming}{Array Programming}
\label{lab:Python_Vectorization}

NumPy allows Python to behave like an array programming language.
It does this by allowing operations that normally apply only to scalars to also work seamlessly with NumPy arrays.
Some of the benefits of allowing these types of operations are concise code and fast execution.
Array programming encourages operations on entire arrays without explicitly looping through them.
In Python this is desirable because explicit for-loops are generally slow.
This lab is some fun exercises using array programming!

\section*{Shuffle}
This exercise is to write a program that shuffles a deck of a cards how a human 
would shuffle them opposed to have a random number generator put them in a random arrangement. 
In order to this, first you cut the deck in half. 
For the first card in each half the either the card in the first half goes down then 
the card in the second half or vice-versa. Which one happens is random. 
Then you follow the procedure for the rest of the cards in the two halves. 
The following is the code implemented using a loop. 
\begin{lstlisting}
from random import randint
def shuffle(deck):
    size = deck.size
    newdeck=np.empty_like(deck)
    for i in xrange(size/2):
         if randint(0,1) == 0:In Python you can open images
            newdeck[i*2] = deck[i]
            newdeck[i*2+1] = deck[size/2+i]
         else:
            newdeck[i*2] = deck[size/2+i]
            newdeck[i*2+1] = deck[i]
    return newdeck
\end{lstlisting}

\begin{problem}
Write a function that shuffles a deck similar to the function above using only array operations.
\end{problem}

\section*{Image Editor}
An image is a 3D array where the first and second dimensions are the height and width respectively 
and the third dimension represents the intensity of each of three color channels: red, green, and blue.
The code below can be used to read and display images in Python.
\begin{lstlisting}
import matplotlib.pyplot as plt
img = plt.imread(<image-file>)
plt.imshow(img)
plt.show()
\end{lstlisting}

\begin{problem}
Write a function that is an image editor that inverts an image, changes it to grayscale, 
or adds a motion blur (definitions below). 
The function should take in a image, a parameter that tells how to modify the image 
and a optional parameter for the motion blur. It should then plot the image. 
\end{problem}

Invert
Every color value for every pixel is changed to its inverse value. For example, 0 
becomes 255, 230 becomes 25, and 127 becomes 128. Remember that the minimum 
color value is 0 and the maximum is 255.

Grayscale
To convert an image to grayscale, each pixel’s color value is changed to the average of 
the pixel’s red, green, and blue value. For example:

Original pixel color values:
Red: 225 Green: 30 Blue: 131
Grayscale coversion: (225 + 30 + 131) / 3 = 128 (using integer division)
Red: 128 Green: 128 Blue: 128

Motion blur
An additional parameter will be used if we use motion blur. We will call this number n. n must be greater than 0.
The value of each color of each pixel is the average of that color value for n pixels (from 
the current pixel to n-1) horizontally. So pixel [x,y,0] would turn in to the average of
pixel[ x, y,0 ] to pixel[ x,y+n-1,0] 
Example: if we store the pixels in a 2d array, the motion blur would average each color 
from Note: You will need to use a for loop here.
Be sure to account for the situations where one or more of the values used in the 
computing the average do not exist. For example, if an image has width w and we are 
considering the pixel on row r, column c, if c + n >= w, then we only average the pixels 
up to w. (For this problem do not worry too much about this case)

\section*{Using \texttt{numpy.where}}
The \li{numpy.where} function takes in a condition and changes the elements in the 
vector that fits the condition to one thing and the elements that do not to another. 
For example:
\begin{lstlisting}
x = np.array([-2, -1, 0, 1, 2])
a = np.where(x < .5, -1, 4)
\end{lstlisting}
is the same as
\begin{lstlisting}
x = np.array([-2, -1, 0, 1, 2])
size = len(x)
a = np.zeros(size)
for i in xrange(size):
    if x[i] < .5:
        a[i] = -1
    else:
        a[i] = 4
\end{lstlisting}

\begin{problem}
Use the \li{np.where} to change any intensity value on an image below 128 to 0 and 128 and above to 255 then plot it.
\end{problem}
