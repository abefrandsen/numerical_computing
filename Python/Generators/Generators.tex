\lab{Python}{Generators}{Generators}
\label{lab:Python_Generators}

Lists, tuples, sets, and dictionaries are examples of sequences in Python.
Often it is useful to visit all the elements of a sequence.  This process of visiting
each element of a sequence once is called \emph{iteration}. 
Python types above each define their own iterators.  You use them every time you execute the statement \li{for <elem> in <list>}.
Python has a special type of object called a \emph{generator}.
Similar to an iterator, a generator returns a sequence of values.
The important difference, though, is that a generator computes the next value in the sequence and returns it.
One way to think of it is: \emph{Iterators return their values while generators yield their values}.
In fact, Python uses the \li{yield} keyword to define a generator.
Let's illustrate the difference between iterating and generating by re-implementing Python's \li{range} function
as both an iterator and a generator.
\begin{lstlisting}
def range_iter(start, stop, step=1):
    i = start
    r = []
    while i < stop:
       r.append(i)
       i = i + step
    return r
\end{lstlisting}
\begin{lstlisting}
def range_gen(start, stop, step=1):
    i = start
    while i < stop:
        yield i
        i = i + step
\end{lstlisting}
The two functions look very similar.  But, you will soon see that they behave very differently.
The first function, when executed will immediately build a list, one element at a time until the done
at which point it returns the entire list.  On the authors computer, this function takes about 1.43ms
to build and return a list of 10000 elements.  The second function is only marginally better at
returning the 10000 elements in approximately 1.09ms.  
This gap between the two function understandably grows much wider as the inputs are increased.
Where the real difference in the two functions lies is in the time the function takes to execute when
it is first called.  The first function requires 1.43ms to execute the first time.  The second function,
a mere .00000041ms!  The reason for this is the first function calculates its results and returns them all at once.  The second function only creates a \emph{generator} object.  This object has several methods.
The most important and most useful method is the \li{next()} method.  Every time this method is called, 
the generator is resumed from is previous state (each time \li{yield} statement is executed, the generator 
is effectively suspended until \li{next()} is called again).  When the generator has finished executing, a
\li{StopIteration} exception is raised the generator terminates.  Capability has been added to send values 
and throw exceptions to the generators via the \li{send()} and \li{throw()} methods respectively.  For more information on these methods, we refer you to PEP 342 (Python Enhancement Proposal 342).

How are generators effective?  If you encounter the situations below, you consider trying to solve 
your problem using generators.  Generators have proven to be very useful in these situations.
\begin{itemize}
\item \emph{Iterating through only part of a sequence.}  Sometimes it is inefficient to create the 
entire sequence if we know that we will only sometimes use all of it.  If we could write a generator for 
the sequence, we would only need to calculate the values needed from the sequence.
\item \emph{Iterating through a sequence once.} Consider the statement 
\li{sum([i for i in range(1000) if i\%2 == 0])}.
We are creating two sequences just to iterate through them once and never use them again.
We can more efficiently write the same statement using generators.
\li{sum(i for i in xrange(1000) if i\%2 == 0)}
Often the solution using generators will execute faster, but it will almost always be more memory efficient.  Consider using generators for any function that reduces a sequence to a single value.  Examples of such functions are \li{min()}, \li{max()}, and \li{sum()}.
\item \emph{Calculating large sequences.}  The sequence must be stored somewhere in computer memory.
If the sequence is very large, we could very easily exhaust all available memory.
\item \emph{Calculations involving infinite sequences.}  Sequences that are pre-computed are, by physical limitations, finite.  We cannot create a list that stores all natural numbers.  However, we can create a generator that returns the next natural number every time it is called.
\end{itemize}


\section*{Combinations}
Combinations are subsets of a set.  The powerset of a set, $S$, is the set of all combinations
of elements in $S$.  The cardinality of the the powerset is $2^{\abs{A}}$.  Clearly, if our
set is of any appreciable size, the powerset will much larger.  Let's look at one method for
generating the powerset of a set.

We can generate the powerset of a set by representing each element in a set by one bit.
If the bit is 0, then we do not include the element in the current combination.  If the
bit is 1, the element is included in the combination.  Then we count from 0 to $2^{\abs{A}}$
and look as the binary representation of the number, including or excluding elements based
on the bits in their respective location.

\begin{problem}
Write a function that will accept a set of elements and return a list of all the combinations
of elements of that set.  You will have to impose some sort of ordering on the set (map each element
in the set to a position in the binary representation).
\end{problem}

This is however an inefficient way to generate combinations.  Ideally, we would want
to generate the next combination by modifying the previous combination.  However, with
out current method of counting in binary, we are rebuilding the combination from scratch
each time.  For example it we have a combination represented by \texttt{01111}, the next
combination would be \texttt{10000}.  We added one element and removed four elements!
We want to get our next combination by changing the previous combination as little as possible.
Fortunately, there is a really neat method for doing this.  We now introduce you to
Gray codes.  Gray codes is a reflected binary code where each new code is generated by
changing the previous code by exactly one bit.  Geometrically, we can think of it as
traversing a unit cube by moving only along the edges of the cube.  Grey codes are used frequently
in error correction schemes.

A Gray code is constructed as follows.
\footnote{Brualdi, Richard Brualdi. \emph{Introductory Combinatorics}. New Jersey: Pearson, 2009}
To calculate the Gray code of order $n$:
\begin{enumerate}
\item The Gray code of order 1 is 0, 1.
\item Compute the Gray code of order $n-1$.  Write the codes and then write them again in reverse order (reflect them so that the last Gray code or order $n-1$ is first and the first Gray code of order $n-1$ is last).
\item Prepend a 0 to the first $n-1$ codes and prepend a 1 to the remaining $n-1$ codes.
\end{enumerate}
While this is a simple algorithm to describe, it is not very efficient.  
To generate a Gray code of order $n$, we have to generate all previous Gray codes.  
If we were to calculate the Gray codes up to order 6, we would calculate the Gray code of order 6 once, order 5 twice, order 4 three times, order 3 four times, and order 2 five times!
Fortunately, there exists another way to compute Gray codes of order $n$ much more efficiently.
The algorithm is given in Brualdi's book.  
We note first that a Gray code of order $n$ is of length $2n-2$ with each code of length $n$.
We also observe that the reflected gray codes always begin with $0\dots0$ and terminate with $10\dots0$.
\begin{enumerate}
\item Start with $0\dots0$ ($n$ zeros).  This is the current Gray code.
\item Sum the digits of the current Gray code.
\begin{enumerate}
\item If the sum of the digits is even, add 1 to the last digit mod 2.
This becomes our new current Gray and we go back to step 1.
\item If the sum of the digits is odd, locate digit $i$, where $i$ is the rightmost non-zero digit.  Add 1 to the $i-1$ digit mod 2.  This becomes our new current Gray and we go back to step 1.
\end{enumerate}
\end{enumerate}

\begin{problem}
\label{prob:brualdi_gray}
Implement the algorithm above for calculating the Gray code of order $n$.
Your implementation must function as a generator.
\end{problem}

\begin{problem}
Each successive gray code only differs from the previous gray code by a single element.
Write a generator yields only the changes between subsets.  Each time the generator is called, 
it should return a single element and a boolean value that determines if that element should be
added or removed from the set.  This function should be able to accept any set with arbitrary elements.
\end{problem}

\section*{Itertools}
There is a very powerful library in the Python Standard Library that is built around the concept
of generators and iterators.  The functions in \li{itertools} are designed to be used as
building blocks.  They are fast and memory efficient.
We encourage you to view the documentation for \li{itertools}.
The documentation also includes various recipes for very useful generator functions.

\begin{problem}
\label{prob:subblocks}
Write a generator function that will evenly split a 1-dimensional array into sub-blocks.
Your function should be capable of returning sub-blocks that could possibly overlap.
The function should accept as arguments: a 1-dimensional array,
a block width, and an optional offset.  If array cannot be evenly divided into
sub-blocks, raise and error.
\end{problem}

\begin{problem}
Expand your solution to Problem \ref{prob:subblocks} to work with 2-dimensional and 3-dimensional
arrays.  You may implement them as separate functions if you wish.
\end{problem}

\printbibliography

% \begin{thebibliography}{99}
% \bibitem{brualdi09}
% Richard Brualdi, 
% \emph{Introductory Combinatorics}.
% Pearson, New Jersey, 5th edition, 2009.
% 
% \end{thebibliography}
