\lab{Algorithms}{Cholesky Decomposition}{Cholesky Decomposition}

\objective{Understand and implement the Cholesky Decomposition}

The Cholesky Decomposition expresses a positive definite matrix as the product of an upper-triangular matrix and its adjoint. This factorization is twice as fast as the LU decomposition and is also more numerically stable. This factorization is useful in a variety of applications, including least squares, optimization, and state estimation. The usefulness is often derived from its numerical stability.

As stated above the Cholesky Decomposition expresses a positive definite matrix $A$ as:
\[
A = LL^H
\]
Where L is lower triangular. A particular algorithm for finding L is defined as follows:

\[
L_{i,j} = \frac{1}{L_{j,j}}\left(A_{i,j} -\sum_{k=1}^{j-1}{L_{i,k}L_{j,k}^*}\right) \mbox{ for i \textgreater  j}
\]
\[
L_{i,i} = \sqrt{A_{i,i} - \sum_{k=1}^{i-1}{L_{i,k}L_{i,k}^*}}
\]

Note that this definition is self-referencing: later solutions requre earlier ones. Thus to implement this formula you have to start in the upper left-hand corner and work your way down.

It is important to note that the Cholesky decomposition is only applicable to positive definite matrices. These matrices are always symmetric, and only have positive eigenvalues. While this characterization is clearly less general than the LU factorization, positive definite matrices are common in a variety of applications, so this method is important to understand.

\begin{problem}
Write the Cholesky decomposition in psuedocode, so that a programmer could code it up for you.
\end{problem}

\begin{problem}
Write your own Cholesky decomposition function. Test it using a random symmetric matrix (build a random square matrix A and then use A'A). Check the output of your function to ensure that it is functioning properly. Compare how long your function takes in comparison to the LU function that you wrote eariler.
\end{problem}
