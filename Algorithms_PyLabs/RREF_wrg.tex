\lab{Algorithms}{Reduced Row Echelon Form}{Reduced Row Echelon Form}
\objective{Write your own code to put a matrix in Reduced Row Echelon Form}

Recall that there are three elementary row operations:

\begin{description}
\item[Type I]  Swap two rows: $R_j \longleftrightarrow R_k$
\item[Type II] Multiply a row by a constant:  $a* R_j \longrightarrow R_j$
\item[Type III] Add a multiple of one row to another: $ a R_k + R_j \longrightarrow R_j$
\end{description}
By combining these operations, we can perform row reduction. Type
the following three m-files.  Be sure the path is set to the correct
directory and that the file names correspond to the function names.

\begin{verbatim}
function A = type1(A,j,k)
%
% This function swaps the j-th and k-th rows
%

temp = A(j,:); A(j,:) = A(k,:); A(k,:) = temp;
\end{verbatim}

\begin{verbatim}
function A = type2(A,j,a)
%
% This function multiplies the j-th row by the constant a
%

A(j,:) = a*A(j,:);
\end{verbatim}

\begin{verbatim}
function A = type3(A,j,k,a)
%
% This function multiplies the constant a times
% the k-th row and adds it to the j-th row
%

A(j,:) = a*A(k,:) + A(j,:);
\end{verbatim}
Now test your functions by running the following:
\begin{verbatim}
>> A = [4 5 6 3; 2 4 6 4; 7 8 0 5]

>> B = type1(A,1,2)
\end{verbatim}
This swaps the first two rows. Now continue with the other two
functions:
\begin{verbatim}
>> C = type2(B,1,0.5)

>> D = type3(C,2,1,-4)

>> E = type3(D,3,1,-7)
\end{verbatim}
At this point we've reduced the first column of the system.  Let's
continue with the second column:
\begin{verbatim}
>> F = type2(E,2,-1/3)

>> G = type3(F,1,2,-2)

>> H = type3(G,3,2,6)
\end{verbatim}
We're almost done.  Let's finish the reduce the third column.
\begin{verbatim}
>> I = type2(H,3,-1/9)

>> J = type3(I,2,3,-2)

>> K = type3(J,1,3,1)
\end{verbatim}
We have successfully row reduced the matrix.  We can check our work
by typing {\tt rref(A)}.

\section*{Programming Row Reduction}

When we do row reduction by hand, we usually take extra steps to
avoid fractions.  However, computers aren't slowed down by decimal
computations.  Below we will re-do the above reduction, but a little
more efficiently.  In addition, rather than defining a new matrix in
every step, we will redefine the matrix $A$ to be the output from
the last calculation.
\begin{verbatim}
>> A = [4 5 6 3; 2 4 6 4; 7 8 0 5]

>> A = type3(A,2,1,-A(2,1)/A(1,1))

>> A = type3(A,3,1,-A(3,1)/A(1,1))
\end{verbatim}
Notice that we didn't row swap or multiply a row by a constant to
make the leading entry a one.  Let's continue:
\begin{verbatim}
>> A = type3(A,3,2,-A(3,2)/A(2,2))

>> A = type3(A,1,2,-A(1,2)/A(2,2))
\end{verbatim}
The second column is reduced.  Now for the third column:
\begin{verbatim}
>> A = type3(A,2,3,-A(2,3)/A(3,3))

>> A = type3(A,1,3,-A(1,3)/A(3,3))
\end{verbatim}
The final step is to divide each row by its leading coefficient:
\begin{verbatim}
>> A = type2(A,1,1/A(1,1))

>> A = type2(A,2,1/A(2,2))

>> A = type2(A,3,1/A(3,3))
\end{verbatim}
We are finished.  Compare your answer with what you get from {\tt
rref}.  We can generalize this process by writing a function:
\begin{verbatim}
function A = rref_3x4(A)
%
% Reduces a 3 x 4 augmented matrix into rref
%
A = type3(A,2,1,-A(2,1)/A(1,1)); A = type3(A,3,1,-A(3,1)/A(1,1)); A
= type3(A,3,2,-A(3,2)/A(2,2)); A = type3(A,1,2,-A(1,2)/A(2,2)); A =
type3(A,2,3,-A(2,3)/A(3,3)); A = type3(A,1,3,-A(1,3)/A(3,3)); A =
type2(A,1,1/A(1,1)); A = type2(A,2,1/A(2,2)); A =
type2(A,3,1/A(3,3));

\end{verbatim}
Now test your function and compare with {\tt rref}.
\begin{verbatim}
>> A = randn(3,4)

>> rref_3x4(A)

>> ref(A)
\end{verbatim}

\section*{Remarks}

\begin{itemize}
\item The above algorithm won't work on every matrix because there is
no logic for dealing with under-determined systems. As a result,
it's a good idea to test this algorithm with random matrices. Why is
that?
\item Fast computer algorithms don't really compute the RREF this
way. They actually compute the REF and then use back-substitution to
finish.
\item The above algorithm is actually called ``Row Reduction Without
Pivoting''. It turns out that the swap operation is important in
dealing with certain matrices that are ``almost'' under determined.
\end{itemize}

\begin{problem}
Write a MATLAB function called {\tt myrref}, which takes as
input an $n\times (n+1)$ matrix and performs the above naive row
reduction. Compare your answers with MATLAB's {\tt rref}. HINT:
Write a nested {\tt for} loop and type {\tt help continue} to get a
good idea of what to do when $j=k$.
\end{problem}
