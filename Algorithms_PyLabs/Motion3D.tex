\lab{Algorithms}{Motion in 3D}{Motion in 3 Dimensions}

\objective{How to use MATLAB to make neat things move in 3d.}

{\bf Outline:}

\begin{itemize}
\item One use of graphics handles is to allow simulated motion in 3D. This can be particularly useful in graphics applications.
\item Walkthrough building an object and animating it. I think a good object might be attached pendulums (you get cool, non-trivial, chaotic sorts of motion). Following Dr. Beard's method for the helicopter is probably a good idea.
\item I think that this walkthrough will probably take most of the space. The idea really is to give them a good idea of how it could be done...
\end{itemize}

\begin{problem}
Model the solar system and make it move. Maybe  just the inner planets is a good idea. Use elliptical orbits, with accurate orbital speeds (maybe give a simplified table).
\end{problem}