\lab{Algorithms}{Image Compression}{Discrete Cosine Transform}

\objective{Understand DCT and build your own image compressor.}

In today's world of high definition entertainment, it is important to be able to efficiently store and retrieve large amounts of data.

At the heart of JPEG compression is the discrete cosine transform. 

DCT is defined by the following equation
\begin{equation}
 P_n(t)=\sqrt\frac{1}{n}y_0+\sqrt\frac{2}{n}\sum_{k=1}^{n-1}y_k\cos\frac{k(2t+1)\pi}{2n}
\label{eqn:dct}
\end{equation}

Or we can express it in terms more familiar in linear algebra.
\[
Cx=y 
\]
We define $x$ to the input vector and $y$ is the result.  We also define $C$ (also known as the \emph{DCT matrix}) to be the following matrix.

\[
C=\sqrt{\frac{2}{n}}\begin{pmatrix}
 \frac{1}{\sqrt{2}} & \frac{1}{\sqrt{2}} & \cdots & \frac{1}{\sqrt{2}} \\
\cos\frac{\pi}{2n} & \cos\frac{3\pi}{2n} & \cdots & \cos\frac{(2n-1)\pi}{2n} \\
\cos\frac{2\pi}{2n} & \cos\frac{6\pi}{2n} & \cdots & \cos\frac{2(2n-1)\pi}{2n} \\
\vdots & \vdots & \cdots & \vdots \\
\cos\frac{(n-1)\pi}{2n} & \cos\frac{(n-1)3\pi}{2n} & \cdots & \cos\frac{(n-1)(2n-1)\pi}{2n}
\end{pmatrix}\]

Notice how this equation is very similar to $Ax=b$ except in this case, instead of solving for $x$, we are given $x$ and want the result.  This is called the \emph{forward DCT}.  The reverse process, or \emph{inverse DCT}, we're back to solving $x = C^{-1}y$.  What would be the most efficient way to calculate the $C^{-1}$?  The answer might be a surprise.

To construct $C$ in Python, we use the following function.  The strategy is to build $C$ by columns.  Our results are a giant column vector which we need to reshape so it is $n \times n$ and then return the transpose.
\begin{lstlisting}
: import scipy as sp
: def dctmat(n):
....: return sp.array(sp.cos([(j+1.0/2)*k*sp.pi/n for j in range(n) for k in range(n)]).reshape((n,n))).T
: A = dctmat(8)
\end{lstlisting}

One of the special properties of \li{A} is that it is orthogonal.  Let's verify that this is in fact the case (it will check the validity of our array generation function).  The \li{sp.allclose()} method allows us to check if an array is within certain tolerances.  We use the method check if $A^TA$ is the identity array.  Let's go ahead and normalize each of the row vectors.
\begin{lstlisting}
: from scipy import linalg as la
: A = sp.array([i/la.norm(i) for in a])
: sp.allclose(sp.dot(la.inv(A), A), sp.eye(8))
True
: A
array([[ 0.35355339,  0.35355339,  0.35355339,  0.35355339,  0.35355339,
         0.35355339,  0.35355339,  0.35355339],
       [ 0.49039264,  0.41573481,  0.27778512,  0.09754516, -0.09754516,
        -0.27778512, -0.41573481, -0.49039264],
       [ 0.46193977,  0.19134172, -0.19134172, -0.46193977, -0.46193977,
        -0.19134172,  0.19134172,  0.46193977],
       [ 0.41573481, -0.09754516, -0.49039264, -0.27778512,  0.27778512,
         0.49039264,  0.09754516, -0.41573481],
       [ 0.35355339, -0.35355339, -0.35355339,  0.35355339,  0.35355339,
        -0.35355339, -0.35355339,  0.35355339],
       [ 0.27778512, -0.49039264,  0.09754516,  0.41573481, -0.41573481,
        -0.09754516,  0.49039264, -0.27778512],
       [ 0.19134172, -0.46193977,  0.46193977, -0.19134172, -0.19134172,
         0.46193977, -0.46193977,  0.19134172],
       [ 0.09754516, -0.27778512,  0.41573481, -0.49039264,  0.49039264,
        -0.41573481,  0.27778512, -0.09754516]])
\end{lstlisting}

Now that we have a usable DCT array, let's create some data to use it on.
\begin{lstlisting}
: data = sp.array([168,154,143,148,120,108,95,100])
: data2 = sp.array([[251, 226, 140, 220, 225,  80,  39,  79]])
: y = sp.dot(A, data); y
array([ 366.28131265,   70.32879668,   -0.38268343,   -2.42692335,
         12.72792206,    5.01087736,    0.92387953,   -8.93652664])
\end{lstlisting}

The values in $y$ are the coefficients to the $\cos$ functions in our DCT equation.  

The basic idea of DCT is translating data stored in a spatial domain to a frequency domain.

DCT formula.

One dimensional case

\begin{problem}
\label{prob:1ddct}
Implement a function that will calculate the 1D DCT of an arbitrary input vector
\end{problem}


Two dimensional case
\begin{problem}
\label{prob:2ddct}
Extend Problem \ref{prob:1ddct} to 2 dimensions.
\end{problem}

\begin{problem}
Modify your solution to Problem \ref{prob:2ddct} to accept an image as input.  Your script should now be a fully working program with functions for opening images from filenames passed in from the commandline as arguments
\end{problem}
