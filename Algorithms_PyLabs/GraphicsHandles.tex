\lab{Algorithms}{Graphics Handles}{Graphics Handles}

\objective{Understand the the use of graphics handles in applications.}

{\bf Outline}

\begin{itemize}
\item Explain that the plot tool is a little unwieldy. We can use plot handles to adjust plots directly.
\item Give examples of being able to modify specific data sets on a plot, and change plot properties with ease. Include: YData, text, mesh vs. surf, scale, axes etc.
\item Essentially the point is to be able to modify plots directly.
\item Introduce MATLAB GUI's. This will require a little bit of work, but could be pretty cool. Uses the idea of handles to manipulate user input.
\end{itemize}

\begin{problem}
Write a script that changes every time the user hits enter. Maybe the logistic or horseshoe map (that could be a fun intro to chaos). Or, less interestingly, have sine with different frequencies.
\end{problem}

\begin{problem}
Make a gui that changes the frequency of sine based upon a slider. Now add a set of checkboxes that allows us to show sine and/or cosine.
\end{problem}

\begin{problem}
Implement a gui on the Julia Sets problems, allowing modification of parameters.
\end{problem}