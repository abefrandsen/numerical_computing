\lab{Applications}{Riemann Sphere and Mobius Transformations}{Riemann Sphere and Mobius Transformations}

\objective{Understand the Riemann Sphere in graphics applications.}

We have now examined several applications of complex numbers and functions.  In this lab we extend several of these ideas and develop some intuition about the Riemann sphere using visualization techniques in Python.

Recall that the complex numbers are an extension of the real numbers that include the imaginary numbers.  We extend them even further in this section and examine the extended complex numbers.  Similar to the extended real numbers, the extended complex numbers are our regular set of complex numbers with the addition of positive and negative infinity.  This allows us to examine, for example, certain quotients that are undefined on the standard complex numbers.

The Riemann sphere construction allows a compact and intuitive construction of the extended complex numbers.  Consider the standard complex plane, only now in 3-space instead of a 2 dimensional plane.  Let the plane pass threw the z-axis at $0$.

{\bf PICTURE HERE}

Now consider a the unit sphere combined with the complex numbers in 3-space:

{\bf PICTURE HERE}

We say that the point at the top of the sphere is positive infinity, while the point at the bottom of the sphere is negative infinity.  We are able to construct any point on the complex plane as the point where a line going from either positive or negative infinity and intersecting the sphere at one point.  Such a line can be constructed to intersect the complex plane at any point of our choosing.  Here are some examples:

{\bf PICTURE AND GOOD CAPTION HERE}.

How do we construct such a line?  Using our understanding of lines in 3 space the construction is rather straightforward.

\begin{problem}  Find the point on the Riemann sphere that along with the point at infinity, creates the line that intersects the complex plane at $1 + 5i$, $10 + 15i$, and $1000 + 1500i$.  Why do we say that the top and bottom of the sphere are infinities?  Draw these lines and the Riemann sphere in Python.  Write a function that, given a point on the complex plane, finds the corresponding point on the Riemann sphere and visualizes it.
\end{problem}

\section*{Mobius Transformations}

Recall from chapter {\bf ???} that a conformal mapping is a complex function that preserves angles between lines.  We now examine a special kind of conformal mapping called a Mobius transformation.  We first give the definition of the Mobius transform, and then examine what sorts of transformations we can do with them.

\begin{definition}  A Mobius transformation is a any complex function of the form
\[
f(z) = \frac{az + b}{cz + d}
\]
With the restriction that $ad \neq bc$
\end{definition}

The Mobius transformation is versatile enough to include translations, rotations, magnifications, or any combination of the same, of shapes in the complex plane.  For example, if we wish to translate the unit disk on the complex plane and then magnify it two times, we would use the following transformation:
\[
PUT IT HERE
\]

{\bf PICTURE GOES HERE}

\begin{problem} Write a Python function that accepts arguments $a,b,c,$ and $d$.  Perform the corresponding Mobius transformation on a grid inside the box $[-1,1]\times[-i,i]$, and then visualize the box before and after the transformation on the same plot, but using different colors.
\end{problem}

What do these special transformations have to do with the Riemann sphere?  Recall from the conformal maps chapter that we can construct any point on the complex plane by drawing a line from infinity through a point on the Riemann sphere.  Consider the points on the sphere that can be used to construct the grid in the previous problem.

{\bf PICTURE GOES HERE}

It turns out that we can express any Mobius transformation as a translation or rotation of the Riemann sphere.  For example:

{\bf SEVERAL PICTURES WITH GOOD CAPTIONS GO HERE}

EXAMPLE OF HOW TO FIND THE TRANSFORMATION OF THE SPHERE HERE

\begin{problem}  Write a Python program that extends the previous problem to 3-dimensions.  Visualize the Mobius translation as a rotation/translation of the Riemann sphere.
\end{problem}



