\lab{Applications}{Norms and Geometry}{Norms and Geometry}
\label{Ch:Norms and Geometry}

\objective{Build intuition about the geometries associated with different norms and metric spaces.}

{\bf Outline}
\begin{itemize}
\item Review $\ell^p$ norms. Talk about how each is equivalent, but yields different notions of geometry.
\item Talk about different matrix norms. Talk about the right way to calculate them as well.
\end{itemize}
\begin{problem}
Write a function plotting unit ball in different norms.
\end{problem}



\begin{problem}
Write a function that transforms some star-shaped object into its equivalent object in other norms. What is the star of david in the infinity norm? (or something like that)
\end{problem}

\section*{Condition Number}

Let $A$ be an invertible $n \times n $ matrix, and consider the system $Ax = b$. Even though $A$ is invertible, we may not be able to compute $x$ exactly.  Maybe $A$ cannot be represented exactly in floating point numbers. Or we may not know $A$ exactly. And even if we can represent $A$ exactly, small roundoff errors can occur in the process of solving the system. Thus, when we solve $Ax-b$ we obtain a solution $x'$ which may be not equivalent to the true solution $x$. Unfortunately, even if $Ax'$ is very close to $b$, $x'$ may not be close to $x$. 

The residual of $x'$ is defined $r = \norm{Ax'-b}$ and the relative residual is $\frac{\norm{r}}{\norm{b}}$. However, we are interested in the relative error of $x'$ which is defined $\frac{\norm{e}}{\norm{x}}$ where $e$ is the error $x-x'$. It is the error, not the residual, which measures how close our imperfect $x'$ is to the true solution $x$. 

Fortunately, these two quantities can be related. If we choose a vector norm $\norm{\cdot} _v$ and a matrix norm $\norm{\cdot} _M$ such that 

\begin{equation} \label{eq:normcondition}
\norm{Ax}_v \leq \norm{A}_M\norm{x}_v \quad \forall x \in  \mathbb{F}^n, A \in \mathbb{F}^{n \times n}
\end{equation}
then it can be shown that 
\begin{equation*}
\frac{1}{\norm{A} \norm{A^{-1}}} \frac{\norm{r}}{\norm{b}} \leq \frac{\norm{e}}{\norm{A}} \leq \norm{A} \norm{A^{-1}} \frac{\norm{r}}{\norm{b}}
\end{equation*}

The number $\norm{A} \norm{A^{-1}} = \kappa (A)$ is called the \emph{condition number} of $A$. If $\kappa (A)$ is small, then $A$ is well=conditioned, and the relative error is close to the relative residual. If $\kappa (A)$ is large, then $A$ is ill-conditioned, and the relative error may be many times larger than the relative residual. Also, if $A$ is ill-conditioned, then small changes in the entries of $A$ can result in large changes in the solutions to $Ax = b$. 

\begin{problem}
Let $A =\begin{pmatrix}1 & 3\\2 & 5.999\end{pmatrix}$ and $b = \begin{pmatrix}5\\9.999\end{pmatrix}$. Calculate the condition number of $A$. Now test the sensitivity of the system $Ax = b$ to small changes in $b$. What do you find?
\end{problem}
